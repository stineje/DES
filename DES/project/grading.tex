\documentclass{article}
\usepackage{amsmath, amsfonts, amsthm, amssymb}  

\usepackage{secdot}
\usepackage{epsfig}
\usepackage{cprotect}
\usepackage[T1]{fontenc}
\usepackage{epstopdf}
\usepackage{url}
\usepackage{rotating}
\usepackage{graphicx}
\usepackage{caption}
\usepackage{subcaption}
\usepackage{multirow}
\usepackage{setspace}
\usepackage{array}
\usepackage{fancyhdr}
\usepackage{lastpage}
\usepackage[T1]{fontenc}

\usepackage{geometry}
\geometry{letterpaper, left=1in, right=1in, top=1in, bottom=1in}

\pagestyle{fancy}
\fancyhf{}
\rhead{\thepage/\pageref{LastPage}}
\lhead{OSU ECEN 3233 - Logic Design - Spring 2021}
\rfoot{\LaTeX}


% ----- Identifying Information -----------------------------------------------
\newcommand{\myassignment}{Final Project Demonstration Grading Rubric }
\newcommand{\myduedate}{Assigned: Wednesday 3/31; Due \textbf{Friday 4/30} (midnight)}
\newcommand{\myinstructor}{Instructor: James E. Stine, Jr.}
% -----------------------------------------------------------------------------

\begin{document}
\begin{center}
  {\huge \myassignment} \\
  \begin{flushright}
  \myinstructor \\
  \end{flushright}
\end{center}

\section{Grading}

For the Spring 2021 Digital Logic Design ECEN3233 class, we will grade
the final 
project via a demonstration in the Endeavor 360 Laboratory on Friday,
April 30, 2021.  The goal of the demonstration is the ability to
showcase the project and what works and how things were accomplished
for the project.

The final demonstration is the single most important measure of the
success of your project. The evaluation is focused on issues of
completion, testing, and reliable operation. You will demo your entire
project to a team of one professor, your TA, and perhaps several peer
reviewers.

Students must be able to demonstrate the full functionality of their
project and any requirements listed in the project description.
Credit will not be given for features which
cannot be demonstrated. For tests that are lengthy or require
equipment not available at the demo, students should have their lab
notebooks or paper ready to show testing data. For any portion of the project
which does not function as specified, students should have hypotheses
(and supporting evidence) of what is causing the problem.

You will have approximately 10 minutes to demonstrate your project to
the evaluators, answer questions, explain detail.  Please try to keep
your time to this time slot.  Do not go over time as we will abruptly
stop the demonstration to go onto the next project demonstration.
Please also arrive early in order to set up your demonstration.  Two
ELVIS III DSDB boards will be set up so that you can set up your
demonstration. 

Specifically, Table~\ref{grade.tbl} is utilized for each project team.
The final scores will be adjusted to be approximately two laboratory
grades for the final project. 
\begin{table} [h!]
  \centering
  {\scriptsize
  \begin{tabular}{|c|l|c|c|c|c|} \hline
    Category & Scoring Criteria & Weight & Evaluation \\ \hline \hline
    Logistics & Students show up on time & 10 & \\ \hline
    Understanding & Questions are answered correctly during the
    demonstration  & 25 & \\ \hline
    Observation & Results that occur during the demonstration are
    properly shown and occur correctly & 25 & \\ \hline
    Completion & The project has been completed entirely & 15 &
    \\ \hline
    Thoroughness & Care and attention to detail are evident in
    construction and layout & 25 & \\ \hline
    Extra Items & Specifics to extra items in the final project & 20 &
    \\ \hline \hline
    Final Score  & & 100 & \\ \hline
  \end{tabular}
  }
  \caption{Final Grading Breakdown for Project}
  \label{grade.tbl}
\end{table}


    
\bibliographystyle{IEEEbib}
\bibliography{lab1}

\end{document}
